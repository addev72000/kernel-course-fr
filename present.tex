%
% This document is available under the Creative Commons Attribution-ShareAlike
% License; additional terms may apply. See
%   * http://creativecommons.org/licenses/by-sa/3.0/
%   * http://creativecommons.org/licenses/by-sa/3.0/legalcode
%
% Created: 2012-03-13 00:05:28+01:00
% Main authors:
%     - Jérôme Pouiller <jezz@sysmic.org>
%

\part{Présentation}

\section{Présentation de la cible}
  % De la board
\begin{frame}{La cible: Calao USB-A9260}
  \begin{center}
    \pgfimage[height=5cm]{pics/USB-AT91-1}
    \hspace{1cm}
    \pgfimage[height=5cm]{pics/USB-AT91-2}
  \end{center}
\end{frame}

\begin{frame}{La cible: Calao USB-A9260}
  Architecture très classique dans le milieu de Linux embarqué:
  \begin{itemize}
  \item Microcontrolleur Atmel AT91SAM9260
  \item Core ARM926EJ-S 180MHz
  \item 64Mo de RAM
  \item 256Mo de flash
  \item 64Ko d'EEPROM
  \end{itemize}
  Choisie car compacte et très bien supportée par Linux
  \note{Montrer: les documentations, la toolchain, le schema de blocks}
\end{frame}

\section{Qu'est-ce que Linux?}

\begin{frame}{Composants de Linux}
  GNU/Linux est finalement un aggloméra:
  \\[2ex]
  \begin{center}
    \begin{tikzpicture}
      \filldraw[cbrown]
       (-0.05,0.95) -- +(9,0) -- +(9,1) -- +(8,1) -- +(8,2) -- +(7,2) -- +(7,3) 
       -- +(6,3) -- +(6,4) -- +(2,4) -- +(2,3) -- +(1,3) -- +(1,1) -- +(+0,1) -- cycle;
% node {Posix};
     \filldraw[ccyan]
       (7,5) -- +(1,0) -- +(1,-1) -- +(1.9,-1) -- +(1.9,0.9) -- +(0,0.9) -- cycle +(1,.5) node {App};
     \filldraw[ccyan]
       (1,5) rectangle +(1.9,0.9) +(1,.5) node {App};
     \filldraw[ccyan]
       (6,4) rectangle +(1.9,0.9) +(1,.5) node {App};
     \filldraw[cyellow]
       (4,4) rectangle +(1.9,0.9) +(1,.5) node {GNU App};
     \filldraw[cyellow]
       (2,4) rectangle +(1.9,0.9) +(1,.5) node {Bash};
     \filldraw[ccyan]
       (1.9,4) -- +(-1,0) -- +(-1,-2) -- +(-1.9,-2) -- +(-1.9,0.9) -- +(0,0.9) -- cycle +(-1,.5) node {App};
     %\filldraw[cyellow]
     %  (0,4) rectangle +(1.9,0.9) +(1,.5) node {App};
     \filldraw[cblue]
       (7,3) -- +(1,0) -- +(1,-1) -- +(1.9,-1) -- +(1.9,0.9) -- +(0,0.9) -- cycle +(1,.5) node {Lib};
     \filldraw[corange]
       (3,3) rectangle +(3.9,0.9) +(2,.5) node {GNU lib};
     \filldraw[corange]
       (1,2) -- +(6.9,0) -- +(6.9,0.9) -- +(1.9,0.9) -- +(1.9,1.9) -- +(0,1.9) -- cycle +(3.5,.5) node {GNU libc};
     \filldraw[cred]
       (0,1) rectangle +(8.9,0.9) +(4.5,.5) node {Noyau Linux};
     \filldraw[cgreen]
       (0,0) rectangle +(8.9,0.9) +(4.5,.5) node {Matériel};
    \end{tikzpicture}
  \end{center}
\end{frame}

\begin{frame}[fragile=singleslide]{Elements}
  Il y a trois éléments important de GNU/Linux:
  \begin{itemize} 
  \item Le noyau: Linux
  \item Le projet GNU et plus particulièrement la glibc, bash et les
    coreutils
  \item Les normes: SystemV, Posix, etc...
  \end{itemize} 
\end{frame} 

\begin{frame}{La Norme Posix}
  \begin{itemize}
  \item \emph{Portable Operating System Interface [for Unix]}
  \item Uniformise les OS
  \item Première version publiée en 1988
  \item Souvent implémenté en partie
  \item ...  et parfois  s'en inspire simplement 
    \note[item]{Parler de µC-OS II, \texttt{printf}, etc...}
  \item Posix $\nrightarrow$ Linux
  \item Linux $\nrightarrow$ Posix
  \end{itemize}
\end{frame}

\begin{frame}{Le Projet GNU}
  \begin{itemize}
  \item Créé en 1983 par Richard Stallman
  \item Pose  les bases politiques  de GNU/Linux
    \note[item]{Expliquer ce qu'est la GPL}
    \begin{itemize}
    \item GPL publiée en 1989
    \item GPLv2 en 1991 
    \item GPLv3 en 2006
    \end{itemize}
  \item \cmd{gcc} apparait en 1985
    \note[item]{Expliquer ce qu'est gcc} 
  \item \cmd{bash} et les  Coreutils apparaissent en 1988 (inspirés de
    \cmd{sh} 1971/1977)
    \note[item]{Expliquer coreutils}
    \note[item]{Expliquer bash} 
  \item Nombre d'architectures supportées incalculable
  \end{itemize}
\end{frame}

\begin{frame}{Le noyau Linux}
  \begin{itemize}
  \item Créé en 1991 par Linus Torvalds:
    \begin{itemize} 
    \item \url{http://groups.google.com/group/comp.os.minix/browse_thread/thread/76536d1fb451ac60}
    \item Inspiré de Minix
    \item \emph{``just a hobby, won't be big and professional like gnu''}
      \note[item]{Parler  de Linus  l'universitaire}
    \end{itemize} 
  \item Noyau monolithique
    \begin{itemize} 
    \item Débat Tanenbaum–Torvalds
    \item \url{http://groups.google.com/group/comp.os.minix/browse_thread/thread/c25870d7a41696d2}
    \item Modulaire depuis  la  version  2.6
      \note[item]{Parler des microkernel}
      \note[item]{Parler de Windows, Mach, Hurd, hyperviseurs}
    \end{itemize} 
  \item Système communautaire
    \begin{itemize} 
    \item La licence GPL à été choisie par pragmatisme et non par conviction
    \item Eternels débats d'opinions entre Stallman et Torvalds
    \end{itemize} 
  \end{itemize}
\end{frame}

\begin{frame}{Le noyau Linux}{Quelques chiffres}
  \begin{itemize}
  \item 15 millions de lignes de code dans 30000 fichiers (+15\%/an)
  \item Environ 1200 développeurs dans 600 entreprises (+35\%/an)
  \item Environ 5000 contributeurs depuis la première version de Linux
  \item  Environ  650  mainteneurs  (c'est-à-dire  responsables  d'une
    partie du noyau)
  \item 26 architecures (= jeux d'instructions)
  \item Des centaines de plateformes
  \item Plus d'un millier de drivers
  \item Une centaine de versions publiées
  \item Environ 10000 contributions sur chaque version
  \item Enormément de ``forks'' et de version non-officielles
  \item Domaine d'application très large, du DSP au super-calculateurs
    en passant pas les grille de calcul
  \end{itemize}
\end{frame}


  % Les choix technologiques historiques:
  %    C vs C++


