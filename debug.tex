%
% This document is available under the Creative Commons Attribution-ShareAlike
% License; additional terms may apply. See
%   * http://creativecommons.org/licenses/by-sa/3.0/
%   * http://creativecommons.org/licenses/by-sa/3.0/legalcode
%
% Created: 2012-03-13 00:09:31+01:00
% Main authors:
%     - Jérôme Pouiller <jezz@sysmic.org>
%

\part{Debuguer}

\begin{frame}
  \partpage
\end{frame}

\begin{frame}
  \tableofcontents[currentpart]
\end{frame}

\section{Afficher des informations}

\begin{frame}[fragile=singleslide]{\texttt{printk}}
  \begin{itemize} 
  \item \c{printk} affiche les information sur la console et les copie
    dans  un  buffer.   Il  est  possible d'afficher  ce  buffer  avec
    \cmd{dmesg}
  \item   Il  est  possible   d'utiliser  directemenr   \c{printk}  en
    specifiant le degré d'importance de l'information:
    \begin{lstlisting} 
printk(KERN_ALERT "Aie\n");
    \end{lstlisting} 
  \item Notez l'absence de virgule entre \c{KERN_ALERT} et la chaîne
  \item Néanmoins,  il est maintenant conseillé  d'utiliser les macros
    \c{pr_cont}, \c{pr_devel}\c{pr_debug}, \c{pr_info}, \c{pr_notice},
    \c{pr_warning},     \c{pr_err},     \c{pr_crit},     \c{pr_alert},
    \c{pr_emerg}
  \item Existe aussi  suffixé de \c{_once} pour n'être  affiché que la
    première fois.
  \item  \c{print_hex_dump_bytes} et  \c{print_hex_dump}  affichent un
    dump en hexadécimal d'un buffer.  La seconde est la forme générale
    qui  permet de  faire à  peu près  la même  chose que  la commande
    \c{hexdump}
  \end{itemize}
\end{frame}

\begin{frame}[fragile=singleslide]{\texttt{printk}}
  \begin{itemize} 
  \item  \c{printk} est  compatible avec  la norme  C99 et  ajoute des
   extensions  pour  pretty-printer  certains pointeurs  (\c{\%p}).  En
   particulier:
     \begin{itemize}
    \item \c{\%pF}  pour résoudre le  nom d'une fonction du  segment de
      text
    \item \c{\%pUB} et \c{\%pUL} pour afficher les UUID ou les GUID
    \item \c{\%pM} pour afficher des adresses MAC
    \item \c{\%pI4} pour affichier des adresses IPv4
    \item \c{\%pI6} et \c{\%pI6c} pour afficher des adresses IPv6
    \item Les  formats d'IP peuvent être suivit  de h, n, l  ou b pour
      demander des conversions d'endian avant d'afficher
    \item \c{\%m} pour \c{errno}
    \end{itemize} 
  \item                Référence                \file{linux/printk.h},
    \file{Documentation/printk-formats.txt}
  \item Il  est aussi possible de  provoquer des dump de  la pile avec
    les macro \c{BUG()}, \c{BUG_ON(condition)} et \c{BUG_ON_NULL(ptr)}
  \item  Les \c{pr_debug}  ne sont  compilé  que si  l'une des  option
    \c{CONFIG_DEBUG} ou \c{CONFIG_DYNAMIC_DEBUG} est active
  \item Dans  le cas de  \c{DYNAMIC_DEBUG}, il est  possible d'activer
    les messages atomiquement en passant par debugfs
  \item Référence: \file{Documentation/dynamic-debug-howto.txt}
  \end{itemize} 
\end{frame} 

\subsection{Dynamic printk}

\begin{frame}[fragile=singleslide]{debugfs}
  \begin{itemize} 
  \item Un certain nom d'outils de debug sont commandé par debugfs
  \item Il s'agit d'un type de partition virtuel
  \item Activé avec 
    \begin{lstlisting} 
mount -t debugfs none /sys/kernel/debug
    \end{lstlisting} 
  \item Il est possible d'obtenir le statut de dprintk avec
    \begin{lstlisting} 
 cat /sys/kernel/debug/dynamic_debug/control
    \end{lstlisting} 
  \item ... et d'activer de nouvelles traces avec 
    \begin{lstlisting} 
echo 'file hello.c +plf' >  /sys/kernel/debug/dynamic_debug/control
    \end{lstlisting} 
  \end{itemize} 
\end{frame} 

%% TODO parler ici des sysrq. Reprendre le docbook de kgdb et de freeelectron
%\begin{frame}[fragile=singleslide]{SysRq}
%\end{frame}

\section{kgdb}

  % kgdb
\begin{frame}[fragile=singleslide]{Debuguer le noyau}
  Activation d \cmd{kgdb} permettant d'embarquer \cmd{gdbserver} dans
  le noyau:
  \begin{itemize}
  \item Compilation
    \begin{lstlisting}
host$ cp -a build build-dbg
host$ make O=build-dbg ARCH=arm CROSS_COMPILE=arm-linux- menuconfig
... Compile the kernel with debug info ...
... KGDB: kernel debugging with remote gdb ...
host$ make -j3 O=build-dbg ARCH=arm CROSS_COMPILE=arm-linux- uImage
    \end{lstlisting} 
  \item L'image ELF est beaucoup plus grosse
    \begin{lstlisting}
host$ ls -l build-dbg/vmlinux build/vmlinux
host$ cp build-dbg/uImage /srv/tftp/uImage-2.6.33.7-dbg
    \end{lstlisting} 
  \end{itemize}
\end{frame}

\begin{frame}[fragile=singleslide]{Debuguer le noyau}
  \begin{itemize}
  \item  Passage des arguments  \cmd{kgdboc} indiquant  quel interface
    \cmd{kgdb} doit utilisé et  \cmd{kgdbwait} indiquant que kgdb doit
    attendre notre connexion avant de continuer le boot
    \begin{lstlisting}
uboot> set bootargs ${bootargs} kgdboc=ttyS0 kgdbwait
uboot> tftp
<Ctrl-a q>
host$ arm-linux-gdb vmlinux 
gdb> target remote /dev/ttyUSB1
    \end{lstlisting} 
    \item Référence: \file{htmldocs/kgdb.html}
  \end{itemize}
\end{frame}

% \begin{frame}[fragile=singleslide]{Debuguer le noyau}
%   Nécessite  une modification  du driver  \file{atmel\_serial}  afin de
%   permettre   l'utilisation  de   son  port   serie   par  \cmd{kgdb}.
%   Modification  apportée par  Albin Tonerre  backportée de  la version
%   2.6.34 du noyau (nous y reviendrons):
%   \begin{lstlisting}
% host$ patch -p1 < 0001-serial-atmel_serial-add-poll_get_char-and-poll_put_c.patch 
%   \end{lstlisting} 

% \note[item]{testé avec la version 2.6.37, mais devait fonctionner comme ca}
% \end{frame}

\begin{frame}[fragile=singleslide]{Commandes gdb utiles}
  Gestion de l'exécution
  \begin{itemize}
  \item Point d'arrêt à l'adresse \verb+0x9876+
    \begin{lstlisting}
gdb> b *0x9876
    \end{lstlisting}
  \item \verb+b  hello.c:30+ Point  d'arrêt à la  ligne 30  du fichier
    \file{hello.c}
    \begin{lstlisting}
gdb> b hello.c:30
    \end{lstlisting}
  \item Point d'arrêt sur la fonction \cmd{main}
    \begin{lstlisting}
gdb> b main
    \end{lstlisting}
  \item Continuer un programme arrêté
    \begin{lstlisting}
gdb> c
    \end{lstlisting}
  \item Démarrer le programme avec \verb+arg+ comme argument
    \begin{lstlisting}
gdb> r arg
    \end{lstlisting}
  \end{itemize}
\end{frame}
  
\begin{frame}[fragile=singleslide]{Commandes gdb utiles}
  Obtenir des informations
  \begin{itemize}
  \item Voir la pile d'appel
    \begin{lstlisting}
gdb> bt
    \end{lstlisting}
  \item Afficher les variable locales à la fonction
    \begin{lstlisting}
gdb> i locales
    \end{lstlisting}
  \item Afficher les arguments de la fonction
    \begin{lstlisting}
gdb> i args
    \end{lstlisting}
  \item Afficher la valeur de \verb+a+
    \begin{lstlisting}
gdb> p a
    \end{lstlisting}
  \item Afficher la valeur de \verb+abs(2 * (a - 4))+
    \begin{lstlisting}
gdb> p abs(2 * (a -  4))
    \end{lstlisting}
  \end{itemize}
\end{frame}

\begin{frame}[fragile=singleslide]{Commandes utiles}
  Debuguer à chaud
  \begin{itemize}
  \item Attacher \verb+gdb+ à un processus existant
    \begin{lstlisting}
gdb> attach 1234
    \end{lstlisting}
  \item Arrêter le debug sans arrêter le processus
    \begin{lstlisting}
gdb> detach
     \end{lstlisting}
   \end{itemize}
\end{frame}

\begin{frame}[fragile=singleslide]{Commandes utiles}
  Et plus...
  \begin{itemize}
  \item Afficher  la valeur de \verb+i+  à chaque fois  que l'on passe
    sur le point d'arrêt 1
    \begin{lstlisting}
gdb> command 1
> silent
> p i
> c
> end
    \end{lstlisting}
  \item Arrêter le programme lorsque la variable \verb+i+ est modifiée
    ou lue
    \begin{lstlisting}
gdb> watch i
gdb> awatch i
      \end{lstlisting}
    \item Passer d'une thread à une autre
      \begin{lstlisting}
gdb> thread 2
gdb> thread 1
      \end{lstlisting}
    \end{itemize}
\end{frame}

\begin{frame}[fragile=singleslide]{Kgdb et les modules}
  \begin{itemize} 
  \item Les modules sont chargé dynamiquement. Ils ne sont pas contenu
    dans le fichier \file{vmlinux}
  \item Il  est nécessaire d'indiquer à  gdb de charger  le module que
    l'on souhaite débuguer
  \item Néanmoins, gdb ne peut pas savoir à quel addresse à été chargé
    le module
  \item    Ces    informations    peuvent   être    récupérées    dans
    \file{/sys/modules/*/sections}
  \item On peut alors les donner à gdb avec \c{add-symbol-file}:
    \begin{lstlisting} 
target% cat /sys/modules/my_modules/sections
host$ gdb vmlinux
> target remote 192.168.1.55:2345
> add-symbol-file my_modules.ko 0xd0832000 \
     -s .bss 0xd0837100 -s .data 0xd0836be0
    \end{lstlisting} 
  \item    Il   existe   des    scripts   effectuant    la   procédure
    automatiquement. A adapter suivant vos besoins.
  \end{itemize}
\end{frame} 

\section{Tracker la mémoire}

\begin{frame}[fragile=singleslide]{kmemleak}
  \begin{itemize} 
  \item Active  un algorithme proche  d'un garbage collector.  Mais au
    lieu  de  libérer  la  mémoire, enregistre  simplement  les  blocs
    perdus.
  \item Le résultat se trouve dans \file{/sys/kernel/debug/kmemleak}
  \item La procédure classique:
    \begin{lstlisting} 
target% echo clear > /sys/kernel/debug/kmemleak
target% modprobe my_module
...
target% modprobe -r my_module
target% echo scan > /sys/kernel/debug/kmemleak
target% cat /sys/kernel/debug/kmemleak
    \end{lstlisting} 
  \item Référence: \file{Documentation/kmemleak.txt}
  \end{itemize} 
\end{frame}

\begin{frame}[fragile=singleslide]{kmemcheck}
  Modifie la  MMU de manière à  ce qu'une exception  soit déclenchée à
  chaque accès à la mémoire. Lors qu'un accès en lecture est effectué,
  kmemcheck  vérifie si  l'octet à  déjà été  écrit auparavent.  Si ca
  n'est pas le cas, une erreur du même type qu'un Oops est générée
  \\[2ex]
  (Pas disponible sur ARM)
  \\[2ex]
  Référence: \file{Documentation/kmemcheck.txt}
\end{frame} 

\section{Outils à base de Kprobe et apparentés}

\subsection{kprobe}

\begin{frame}[fragile=singleslide]{Kprobe}
  D'un point  de vue générique, Les Kprobes  permettent de d'effectuer
  une  action  lors  du  passage  de l'exécution  d'une  case  mémoire
  particulière (Référence: \file{Documentation/kprobes.txt})
  \\[2ex]
  Dans la partie userspace, il  existe une api d'avoir des informations
  statistique sur le passage sur  une ligne.  Il suffit d'indiquer une
  adresse (ou un symbole) à prober :
  \begin{lstlisting} 
echo 'p do_fork' > /sys/kernel/debug/tracing/kprobe_events 
  \end{lstlisting} 
  ... et de l'activer
  \begin{lstlisting} 
echo 1 > /sys/kernel/debug/tracing/events/kprobes/p_do_fork_0/enable             
  \end{lstlisting} 
  On peut maintenant lire les résultats dans:
  \begin{lstlisting} 
cat /sys/kernel/debug/tracing/trace
  \end{lstlisting} 
  Référence: \file{Documentation/trace/kprobetrace.txt}
\end{frame} 

\subsection{Tracepoint}

\begin{frame}[fragile=singleslide]{Tracepoint}
  Les tracepoints sont  des endroit spécifié dans le  code du noyau qui
  peuvent sembler intéressant à mesurer:
  \begin{itemize}  
  \item Pas de possibilité de les déclarer n'importe ou dans le code
  \item Permet d'obtenir plus d'informations que les kprobes
  \item Les développeur du noyau  ont estimé que l'endroit était pour y mettre un tracepoint et ont configuré le format adéquate (cf.
    \file{/sys/kernel/debug/tracing/events/*/format})
  \item  L'overhead pour  un  tracepoint désactivé  est  un poil  plus
    important que  pour les kprobe  vu qu'il n'y  a pas de patch  à la
    volée du code (mais du coup l'usage est permit en XIP)
  \end{itemize} 
  Pour en obtenir la liste des tracepoints:
  \begin{lstlisting} 
find /sys/kernel/debug/tracing/events/ -type d
  \end{lstlisting} 
  Il  est évidement  possible  de de  définir  ses propres  tracepoints
  (cf. \file{Documentaiton/trace/tracepoints.txt})
  \\[2ex]
  Référence: \file{Documentation/trace/events.txt}
\end{frame} 

\subsection{Ftrace}

\begin{frame}[fragile=singleslide]{ftrace}
  Permet  d'instrumenter certaines  latence  particulière. Il  utilise
  l'option  \c{-pg} de  \c{gcc}. Cela  ajoute un  appel  à \c{_mcount}
  avant chaque  appel de fonction.  Contrairement à  kprobe, ftrace ne
  peut se placer qu'au début d'une fonction.
  \\[2ex]
  Néanmoins, ftrace est très utilisé dans le temps réel, la mesure des
  temps de latence et l'étude du scheduling.
  \\[2ex]
  Reference:                      \file{Documentation/trace/ftrace.txt}
  \file{Documentation/trace/ftrace-design.txt}
\end{frame} 

\subsection{Oprofile}

\begin{frame}[fragile=singleslide]{oprofile}
  Les CPU  modernes possèdent des  registres de debugs  et breakpoints
  hardware capable de profiler  le code. Oprofile permet de paramétrer
  et  de  visualiser le  contenu  de  ces  registres.  Oprofile  n'est
  disponible que sur certaines architectures.
\end{frame} 

\subsection{Perf}

\begin{frame}[fragile=singleslide]{perf}
  perf vous offre une interface pour accéder :
  \begin{itemize} 
  \item  aux tracepoint
  \item  aux kprobe
  \item  aux registres de debug hardware et au breakpoints hardware
  \end{itemize} 
  Ainsi,  perf offre  une  interface plus  complète  que oprofile  (et
  disponibles sur plus d'architecture)
  \\[2ex]
  (perf ne compile pas sur notre carte de test  car notre toolchain est un peu trop
  vieille pour le noyau 3.3 (et j'ai la flemme de la recompiler))
\end{frame} 

\begin{frame}[fragile=singleslide]{SystemTap}
  Permet    d'accéder   à    quasiment   toutes    les   technologies
  précédentes.  Compiler  en  natif,   ce  qui  permet  d'être  très
  rapide... Nécessite une formation à lui-seul.
  \\[2ex]
  Le principe:
  \begin{itemize} 
  \item  On écrit du pseudo-code C
  \item  Il est converti en un modules C
  \item  Il est compilé, puis chargé sur le noyau de la cible
  \item  Le résultats est récupéré par un service appelé \emph{relayfs}
  \end{itemize} 
  Référence: systemtap-doc
  \\[2ex]
  Pour   une   vue  globale   est   différents   outils  d'instrumentation:
  \file{Documentation/trace/tracepoint-analysis.txt}
\end{frame} 



  % kdb
  % Les magic keys
  % Le soption de debug de HACKING
  % Les autres: systemtap, lttng , ftrace, kernel_probe



